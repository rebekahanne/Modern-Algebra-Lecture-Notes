%% 
%% This is file, `math_541_theorems.tex',
%% generated with the extract package.
%% 
%% Generated on :  2018/12/16,19:59
%% From source  :  "Math 541 Lecture Notes".tex
%% Using options:  active,generate=Math_541_Theorems,extract-env={theorem,corollary}
%% 
\documentclass[12pt]{article}
  \usepackage{amsthm}
  \usepackage{amsfonts, amsmath}
  \usepackage[dvipsnames]{xcolor}
  %\newtheorem{theorem}{Theorem}
  \theoremstyle{definition}
  \definecolor{Tm}{rgb}{0,0,0.80}
  %\newtheorem{definition}{Definition}
  \newtheorem{definition}{\color{NavyBlue}{\textbf{Definition}}}
  \newtheorem{theorem}{\color{ForestGreen}{\textbf{Theorem}}}
  \newtheorem{corollary}{\color{ForestGreen}{\textbf{Corollary}}}
  %\newtheorem{example}{Example}
\newcommand{\e}{\epsilon}
%\newcommand{\d}{\delta}
\newcommand{\D}{\Delta}
\newcommand{\R}{\mathbb{R}}
\newcommand{\Z}{\mathbb{Z}}

\begin{document}

\begin{theorem}[The unit element is unique]
Let $G$ be a group and $\star$ its binary operation. Suppose that $e_1, e_2 \in G$ are both units elements. Then, $e_1 = e_2$.
\end{theorem}

\begin{theorem}[Cancellation Law]
For every group $G$ and $a,b,c \in G$ that satisfy $ab = ac$, we have $b=c$.
\end{theorem}

\begin{theorem}[The inverse of a group element is unique]
Let $G$ be a group and let $a \in G$. If $b$ and $c$ are inverses of $a$, then $b = c$.
\end{theorem}

\begin{theorem}[The order of a $k$-cycle is $k$.]
\end{theorem}

\begin{theorem}[Disjoint cycles commute.]
\end{theorem}

\begin{theorem}[Basic facts about homomorphisms]
Let $\phi : G \to H$ be a homomorphism. Then
\begin{enumerate}
\item $\phi(1_G) = 1_H$ (the identity of $G$ is mapped to the identity of $H$).
\item $\phi(x^{-1}) = \phi(x)^{-1}$ for all $x \in G$.
\end{enumerate}
\end{theorem}

\begin{theorem}[Kernel of a homomorphism is a subgroup]
Let $\phi : G \to H$ be a homomorphism. Then $Im(\phi) = \{ \phi(g) | g \in G \} \leq H$.
\end{theorem}

\begin{theorem}
Let $\phi: G \to H$ be a homomorphism. Then $\ker(\phi) \leq G$. That is, the kernel of $\phi$ is a subgroup of $G$.
\end{theorem}

\begin{theorem}[All left cosets of $H$ have the same size]
Let $H \leq G$ be groups and let $a \in G$. Then $|[a] | = |aH| = |H|$
\end{theorem}

\begin{theorem}[Lagrange]
Let $G$ be a finite group and let $H \leq G$. Then $|H|$ divides $|G|$.
\end{theorem}

\begin{theorem}[Equivalent conditions to be a normal subgroup]
Let $N \leq G$. Then $N \trianglelefteq G$ if one of the following holds:
\begin{enumerate}
\item $\forall g\in G$, $gN = Ng$
\item $\forall g\in G$, $gNg^{-1} = N$
\item $\forall g\in G$, $gNg^{-1} \subseteq N$
\item $\forall g\in G$ and $\forall n\in N$, $gng^{-1} \in N$
\end{enumerate}
\end{theorem}

\begin{theorem}[The kernel of a Homomorphism is a Normal Subgroup]
Let $\phi : G \to H$ be homomorphism. Then $\ker (\phi) \trianglelefteq G$.
\end{theorem}

\begin{theorem}[Every cyclic group is isomorphic to either $\mathbb{Z}$ or to $\mathbb{Z} / n \mathbb{Z}$ for some $n \geq 1$.] For every group $H$ for which there exists an $x \in H$ such that $H = \langle x \rangle$, there exists a bijective homomorphism (i.e. an isomorphism) $\phi : H \to C$ where $C = \mathbb{Z}$ or $C = \mathbb{Z} / n \mathbb{Z}$ for some $n \geq 1$.
\end{theorem}

\begin{theorem}
Let $G$ be a finite cyclic group of order $n$. For every $m | n$ ($m$ that divides $n$) there exists a unique subgroup $H$ of $G$ with $|H|=m$. Furthermore, $H$ is cyclic.
\end{theorem}

\begin{theorem}[Important Identity for Dihedral Groups]
$\rho \e = \e \rho^{-1}$.
\end{theorem}

\begin{theorem}
$\rho^i \e = \e \rho^{-i}$
\end{theorem}

\begin{theorem}
In the above definition, $G/N$ is a group.
\end{theorem}

\begin{theorem}[The First Isomorphism Theorem]
If $\phi:G\to H$ is a homomorphism of groups, then $G / \ker(\phi) \cong Im \phi$.
\end{theorem}

\begin{theorem}[The Second or Diamond Isomorphism Theorem]
Let $H \leq G$ and $K \trianglelefteq G$. Then $HK / K \cong H / H \cap K$.
\end{theorem}

\begin{theorem}[The stabilizer of a group element is a subgroup]
$G_x \leq G$
\end{theorem}

\begin{theorem}[Orbit-Stabilizer Theorem]
There is a bijection
\begin{equation}
f : G / G_x \to O_x
\end{equation}
In words, there is a bijection between the collection of all cosets of the stabilizer and the orbit. In particular,
\begin{equation}
[G:G_x] = |O_x|
\end{equation}
(Recall we defined $|G / G_x|$ to be $[G:G_x]$).
\end{theorem}

\begin{theorem}
An action is transitive if and only if there exists an $x \in X$ such that $O_x = X$. That is, all elements of $X$ have the same equivalence class.
\end{theorem}

\begin{theorem}[Burnside]
Let $G$ act on $X$. Suppose that $G,X$ are finite. Then,
\begin{equation}
N = \text{ \# of orbits (equivalence classes) } = \frac{1}{|G|} \sum_{g\in G} |X^g|
\end{equation}
\end{theorem}

\begin{theorem}
$(\mathbb{Z}/n \mathbb{Z})^{\times}$ is a group.
\end{theorem}

\begin{theorem}[Fermat's Little Theorem]
For a prime number $p$ and $1 \leq x \leq p-1$, we have $x^{p-1} - 1$ is divisible by $p$.
\end{theorem}

\begin{theorem}
Let $p \neq q$ be odd primes. Then
\begin{equation}
(\mathbb{Z}/pq \mathbb{Z})^{\times} \cong (\mathbb{Z}/p \mathbb{Z})^{\times} \times (\mathbb{Z}/q \mathbb{Z})^{\times}
\end{equation}
\end{theorem}

\begin{theorem}
If $\lambda : G \to S_n$ is a homomorphism, we can define an action of $G$ on $\{1,\ldots, n\}$ by
\begin{equation}
g(i) = \lambda(g)(i)
\end{equation}
\end{theorem}

\begin{theorem}
Under the above assumptions: $\lambda$ is injective if and only if the action of $G$ on $X$ (which we can think of as $\{1,\ldots, n\}$) is faithful.
\end{theorem}

\begin{theorem}[Cayley]
Let $G$ be a group of order $n$. Then, there exists an injective homomorphism $\phi : G \to S_n$.
\end{theorem}

\begin{theorem}
If $|G| = n$, then $G$ is isomorphic to a subgroup of $S_n$. Indeed $\phi : G \to Im(\phi) \subset S_n$.
\end{theorem}

\begin{theorem}[Cardinality of set of fixed points of action of set on $p$-group equals cardinality of set mod $p$]\label{lemma:favlemma}
Let $G$ be a $p$-group that acts on a finite set $X$. Let $X^G = \cap_{g \in G} X^g$ where $X^g = \{x\in X| gx =x\}$, that is those $x\in X$ such that for all $g\in G$, $gx = x$. Then $p$ divides $|X| - |X^G|$, that is
\begin{equation}
|X^G| \equiv |X| \quad (mod p)
\end{equation}
\end{theorem}

\begin{theorem}[A $p$-group has a non-trivial center]
Let $G$ be a $p$-group. Then $Z(G) \neq \{1\}$.  In words, there has to be a non-trivial element of the group that commutes with everything else.
\end{theorem}

\begin{corollary}
Let $p$ be a prime number and let $G$ be a group of order $p^2$. Then $G$ is abelian.
\end{corollary}

\begin{theorem}[Cauchy]
Let $G$ be a finite group and suppose that $p \big\vert |G|$ for some prime $p$. Then there exists an element of order $p$ in $G$.
\end{theorem}

\begin{theorem}[Correspondence Theorem]
Let $G, H$ be groups, and let $\phi : G \to H$ be a group homomorphism. Then there exists a correspondence (i.e.\ a bijection)
\begin{equation*}
\{\text{Subgroups $K$ of $G$ containing $\ker \phi$}\} \iff \{\text{Subgroups $L$ of $H$ contained in $Im(\phi)$}\}
\end{equation*}
given by $K \mapsto \phi(K)$ and $L \mapsto \phi^{-1}(L)$. In addition, let $K_1$ and $K_2$ be subgroups of $G$ containing $\ker(\phi)$ and $L_1$ and $L_2$ subgroups $L$ of $H$ contained in $Im(\phi)$.
\begin{enumerate}
\item $K_1 \leq K_2 \implies \phi(K_1) \leq \phi(K_2)$
\item $L_1 \leq L_2 \implies \phi^{-1}(L_1) \leq \phi^{-1}(L_2)$
\end{enumerate}
and
\begin{enumerate}
\item $K_1 \leq K_2 \implies [K_2 : K_1] = [\phi(K_2) : \phi(K_1)]$
\item $L_1 \leq L_2 \implies [L_2 : L_1] = [\phi(L_2) : \phi(L_1)]$
\end{enumerate}
\end{theorem}

\begin{theorem}[Corollary of Correspondence Theorem]
Let $G$ be a group and let $N \trianglelefteq G$. Then the subgroups of $G/N$ are all of the form $R/N$ for some $N\leq R\leq G$. Moreover,
\begin{equation}
[G:R] = [G/N:R/N]
\end{equation}
\end{theorem}

\begin{theorem}[Sylow's Theorem]
Let $p$ be a prime number, let $G$ be a finite group, and let $p^n$ be the largest power of $p$ that divides $
|G|$. Then $G$ contains a subgroup $P$ of order $p^n$. $P$ is called a $p$-Sylow subgroup of $G$.
\end{theorem}

\begin{theorem}[$p$-Sylow subgroups are conjugate]
Let $G$ be a finite group and let $P,Q$ be $p$-Sylow subgroups of $G$. Then there exists $g \in G$ such that $gPg^{-1} = Q$.
\end{theorem}

\begin{corollary}[$p$-Sylow subgroup unique if and only if normal subgroup.]
Let $G$ be a finite group and let $P$ be a $p$-Sylow subgroup of $G$. Then $P$ is a unique $p$-Sylow subgroup if and only if $P \trianglelefteq G$.
\end{corollary}

\begin{theorem}[Sylow's Theorem (General)]
Let $G$ be a finite group and $p$ a prime. Suppose that $p^r$ divides $|G|$. Then $G$ has a subgroup $H$ of order $p^r$. Moreover, every subgroup of order $p^r$ is contained in a Sylow subgroup.
\end{theorem}

\begin{theorem}[$0\cdot a = 0$]
\begin{align*}
0 \cdot a &= (0 + 0) \cdot a \tag{$0$ additive identity}\\
&= 0\cdot a + 0 \cdot a \tag{distributivity}
\end{align*}
Then cancellation gives $0 = 0 \cdot a$.
\end{theorem}

\begin{theorem}[$-a = (-1)\cdot a$]
We want to show that $(-1)\cdot a$ is the additive inverse of $a$. To that end
\begin{align*}
a + (-1)\cdot a &= 1 \cdot a + (-1) \cdot a \tag{$1$ multiplicative identity}\\
&= (1 + -1) \cdot a \tag{distributivity}\\
&= 0 \cdot a \\
&= 0
\end{align*}
\end{theorem}

\begin{theorem}[Homomorphism of rings injective injective if and only its kernel is trivial]
Suppose $\phi$ is a homomorphism of rings. Then $\phi$ is injective if and only if $\ker(\phi) = \{0\}$.
\end{theorem}

\begin{theorem}[Field has only 2 ideals: the trivial ideal and the field itself.]
If $R$ is a field, then its ideals are $R$ and $\{0\}$.
\end{theorem}

\begin{theorem}[There are only 2 ideals in $M_{n\times n}(\R)$]
\end{theorem}

\begin{theorem}[Kernel of homomorphism of rings is an ideal of the ring which is the domain of the homomorphism]
Suppose $A,B$ are rings and let $\phi:A \to B$ be a ring homomorphism. Then $\ker(\phi)$ is an ideal of $A$.
\end{theorem}

\begin{theorem}[Ideal is an additive normal subgroup]
Suppose $A$ is a ring and $I$ is an ideal of $A$. Then $I$ is an an additive normal subgroup $A$.
\end{theorem}

\begin{theorem}
Suppose $\phi : A \to A/I$ is a homomorphism of rings where $\phi(a) = I + a$. Then
\begin{equation}
\ker(\phi) = I
\end{equation}
\end{theorem}

\end{document}
