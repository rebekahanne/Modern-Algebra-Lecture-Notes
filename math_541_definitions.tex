%% 
%% This is file, `math_541_definitions.tex',
%% generated with the extract package.
%% 
%% Generated on :  2018/11/21,3:04
%% From source  :  "Math 541 Lecture Notes".tex
%% Using options:  active,generate=Math_541_Definitions,extract-env={definition}
%% 
\documentclass[12pt]{article}
  \usepackage{amsthm}
  \usepackage{amsfonts, amsmath}
  \usepackage[dvipsnames]{xcolor}
  %\newtheorem{theorem}{Theorem}
  \theoremstyle{definition}
  \definecolor{Tm}{rgb}{0,0,0.80}
  %\newtheorem{definition}{Definition}
  \newtheorem{definition}{\color{NavyBlue}{\textbf{Definition}}}
  \newtheorem{theorem}{\color{ForestGreen}{\textbf{Theorem}}}
  \newtheorem{corollary}{\color{ForestGreen}{\textbf{Corollary}}}
  %\newtheorem{example}{Example}
\newcommand{\e}{\epsilon}
%\newcommand{\d}{\delta}
\newcommand{\D}{\Delta}
\newcommand{\R}{\mathbb{R}}
\newcommand{\Z}{\mathbb{Z}}

\begin{document}

\begin{definition}[Group]
A set $G$ with a binary operation $\star: G \times G \rightarrow G$ is a group if the following axioms are satisfied:
\begin{enumerate}
\item Associativity: $(a \star b) \star c = a \star (b \star c)$ for every $a,b,c \in G$.
\item Unit (or Identity): There exists an $e \in G$ such that $e \star a = a \star e = a$ for each a in $G$.
\item Inverse: For each $a \in G$ there is a $b \in G$ such that $a \star b = b \star a = e$.
\end{enumerate}
\end{definition}

\begin{definition}[Abelian/Commutative]
A group $G$ is abelian or commutative if $a \star b = b \star a$ for all $a \in G$.
\end{definition}

\begin{definition}[The group $\mathbb{Z} / n \mathbb{Z}$]
The group $\mathbb{Z} / n \mathbb{Z}$ is the set $\{0,1,\ldots,n-1\}$. That is, the possible (integer) remainders upon dividing by $n$. Recall that the remainder is the smallest number that you subtract from the original number so that it becomes divisible by $n$.
\end{definition}

\begin{definition}[Order of a group, order of an element of a group]
Let $G$ be a group. We call $|G|$ the order of $G$ (i.e. the number of elements in $G$). Further, the least $d >0$ such that $g^d = 1$ is called the order of $g \in G$.
\end{definition}

\begin{definition}(Cycle, Cycle Decomposition, Length, $k$-Cycle)
A cycle is a string of integers which represents the element of $S_n$ which cyclically permutes these integers (and fixes all other integers). The product of all the cycles is called the cycle decomposition. The length of a cycle is the number of integers which appear in it. A cycle of length $k$ is called a $k$-cycle.
\end{definition}

\begin{definition}[Subgroup]
A subset $H$ of a group $G$ is called a subgroup of $G$ if the following axioms are satisfied
\begin{enumerate}
\item Identity: $1 \in H$ (we could also write $1_G \in H$).
\item Closed under products: $h_1 h_2 \in H$ for all $h_1, h_2 \in H$ (in words, the binary operation of $G$ applied to elements of $H$ keeps products in $H$).
\item Closed under inverses: $h^{-1} \in H$ for all $h\in H$.
\end{enumerate}
In this case we write $H \leq G$. Observe that $H$ is indeed a group.
\end{definition}

\begin{definition}[Homomorphism]
Let $G, H$ be groups. A function $\phi : G \to H$ is a homomorphism if for every $a,b \in G$, we have
\begin{equation}
\phi(ab) = \phi(a)\phi(b)
\end{equation}
Note the the product $ab$ on the left is computed in $G$ and the product $\phi(x)\phi(y)$ is computed in $H$.
\end{definition}

\begin{definition}[Kernel]
Let $\phi:G \to H$ be a homomorphism. Then
\begin{equation}
\ker(\phi) = \{g \in G : \phi(g) = 1 \}
\end{equation}
(note that $1$ is the identity of H).
\end{definition}

\begin{definition}[Coset]
Let $H \leq G$ and fixed $a \in G$. Let
\begin{align*}
aH &= \{ ah | h \in H \} \\
Ha &= \{ ha | h \in H \}
\end{align*}
These sets are called a left coset and right coset of $H$ in $G$.

Write $G/H$ for the set of left cosets $\{aH | a \in G \}$.
\end{definition}

\begin{definition}[Index]
If $G$ is a group (possibly infinite) and $H \leq G$, the number of left cosets of $H$ in $G$ is called the index of $H$ in $G$ and is denoted by $|G : H|$. Alternatively, $|G : H| = | G / H | =  |\{aH | a \in G\} |$. If $G$ is finite, the $|G:H| = \frac{|G|}{|H|}$.
\end{definition}

\begin{definition}[Normal Subgroup]
We say that a subgroup $H$ of $G$ is normal if $aH = Ha$ for every $a \in G$. Write $H \trianglelefteq G$. This means that the left and right cosets of a group of equivalent.
\end{definition}

\begin{definition}[Cyclic Group]
A group $H$ is cyclic if $H$ can be generated by a single element, i.e., there is some element $x \in H$ such that $H = \{x^n | n \in \mathbb{Z} \}$. Write $H  = \langle x \rangle$ and say $H$ is generated by $x$.

An alternative definition is: Let $G$ be a group and fix $x \in G$. Let $H$ be the subset of $G$ that contains all the powers of $x$. Then notice that $H = \{x^n | n \in \mathbb{Z} \}$ is a subgroup of $G$ (the identity element must be in $H$ since $x^0 = 1$, $H$ is closed under products since adding exponents will keep us in $H$, and the inverse of $x^n$ is $x^{-n}$, which is also in $H$). We call $H$ the subgroup of $G$ generated by $x$, $H = \langle x \rangle$, and $H$ is cyclic.
\end{definition}

\begin{definition}[Dihedral Group, $D_n$]
In general, $D_n$ is a group with $2n$ elements, where the binary operation is composition. It contains two types of symmetries:
\begin{enumerate}
\item The rotation $\rho$ is $\frac{2\pi}{n}$ radians clockwise. The set of all rotations is $\langle \rho \rangle = \{1, \rho, \rho^2, \ldots, \rho^{n-1} \}$.
\item Let $\e$ be a vertical mirror symmetry. Then the set of all mirror symmetries is $\{\e, \e\rho, \e\rho^2, \ldots, \e \rho^{n-1} \}$.
\end{enumerate}
\end{definition}

\begin{definition}[Quotient Group]
Let $G$ be a group and $N \trianglelefteq G$ (that is, $N$ is a normal subgroup of $G$). Let $G/N = \{gN|g \in G\}$ be the set of left cosets of $N$ in $G$. Then the quotient group of $G$ by $N$ is the group $(G/N, \cdot)$, where $\cdot$ is the binary operation on $G/N$ defined for all $g_1 N, g_2 N \in G / N$ by $g_1 N g_2 N = g_1g_2N$.
\end{definition}

\begin{definition}[Action]
An action of a group $G$ on $X$ (or we say $G$ acts on $X$) is a function $G \times X \to X$, $(g,x) \to gx$ where
\begin{enumerate}
\item $1_G x = x \quad \forall x \in X$
\item $g(hx) = (gh)x \quad \forall g,h \in G, \forall x \in X$
\end{enumerate}
\end{definition}

\begin{definition}[Orbit]
Given $x \in X$ the orbit of $x$ is
\begin{equation}
O(x) = O_x = \{gx | g \in G\}
\end{equation}
This is the set of all elements that can be reached from $x$ by applying elements from $G$.
\end{definition}

\begin{definition}[Stabilizer, Isotropy Subgroup]
Let $X$ be a $G$-set and $x \in X$. The stabilizer of $x$ is
\begin{equation}
G_x = Stab_G(x) = \{ g\in G | gx = x \}
\end{equation}
also called the isotropy subgroup of $x$.
\end{definition}

\begin{definition}[Transitive action]
We say that an action of $G$ on $X$ is transitive if for every $x,y \in X$, there is an element $g \in G$ such that $gx = y$. In words, this means that we can arrive at $y$ from $x$ by applying an element from $G$.
\end{definition}

\begin{definition}[Action induces equivalence relation]
The action of any group $G$ on $X$ induces an equivalence relation by saying $x \sim y$ if there exists a $g \in G$ such that $gx = y$.
\end{definition}

\begin{definition}[Conjugation action, conjugacy classes, conjugate]
Consider the action of $G$ on itself by $g(x) = gxg^{-1}$. We call this the conjugation action. The equivalence classes created by this action are called the conjugacy classes of $G$. We say that two elements in $x,y\in G$ are conjugate if they belong to the same conjugacy class.
\end{definition}

\begin{definition}[Fixed points]
For any element $g\in G$, let $X^g = \{x\in X|gx = x\}$. In words, this is the set of all elements in $X$ such that $g$ acts on them like the identity.
\end{definition}

\begin{definition}[Free, faithful action]
Let $G$ be a group that acts on a set $X$.
\begin{enumerate}
\item The action is said to be faithful if for all $x \in X$
\begin{equation}
gx = x \implies g = 1
\end{equation}
Thus the only element that acts like the identity is actually the identity $g=1$. Alternatively,
\begin{equation}
\cap_{x \in X} G_x = \{1\}
\end{equation}
\item The action is free if for all $g\in G$ and for all $x \in X$
\begin{equation}
gx = x \implies g = 1
\end{equation}
Alternatively, this means all stabilizers are trivial. We have that for all $x \in X$,
\begin{equation}
G_x = \{1\}
\end{equation}
 Or, any element which has a fixed point is the identity element.
\end{enumerate}
\end{definition}

\begin{definition}[Coprime]
An integer $a$ is coprime to $n$ if the only positive divisor of both $a$ and $n$ is 1.
\end{definition}

\begin{definition}[$(\mathbb{Z}/n \mathbb{Z})^{\times}$]
\begin{equation}
(\mathbb{Z}/n \mathbb{Z})^{\times} = \{1 \leq a \leq n - 1 | a \text{ coprime to } n\}
\end{equation}
$n \geq 2$. This is called the multiplicative group of integers modulo $n$, where the binary operation is multiplication and taking the remainder upon dividing by $n$.
\end{definition}

\begin{definition}[Permutation representation of action]
Let $G$ be a group that acts on $\{1,\ldots, n\}$. Associated to the action is a homomorphism $\lambda : G \to S_n$, defined by
\begin{equation}
\lambda(g)(i) = gi
\end{equation}
where the RHS is the action of $g$ on $i$, $i \in \{1,\ldots, n\}$.
\end{definition}

\begin{definition}[$p$-Group]
Let $p$ be a prime number. $G$ is a $p$-group if $|G|$ is a power of $p$.
\end{definition}

\begin{definition}[Ring]
Let $A$ be a set with two binary operations: addition and multiplication. A is called a ring if:
\begin{enumerate}
\item $A$ is an abelian group under addition:
\begin{enumerate}
\item Addition associative: For all $a,b,c \in A$, $(a+b) + c = a + (b+c)$.
\item Additive identity: There exists a $0 \in A$ such that for all $a \in A$, $a + 0 = 0 + a = a$.
\item Additive inverse: For all $a \in A$, there exists a $b \in A$ such that $a + b = b + a = 0$.
\item Addition commutative: For all $a,b \in A$, $a + b = b + a$.
\end{enumerate}
\item Multiplication associative: For all $a,b,c \in A$, $(a \cdot b) \cdot c = a \cdot (b \cdot c)$.
\item Multiplicative identity: There exists $1 \in A$ such that for all $a \in A$, $1 \cdot a = a \cdot 1 = a$.
\item Multiplication distributive: For all $a,b,c \in A$
\begin{enumerate}
\item $a \cdot (b + c) = a\cdot b + a \cdot c$.
\item $(b + c) \cdot a = b\cdot a + c \cdot a$.
\end{enumerate}
\end{enumerate}
\end{definition}

\begin{definition}[Commutative Ring]
A ring is called commutative if for all $a,b \in A$, $ab = ba$.
\end{definition}

\begin{definition}[Field]
A commutative ring is called a field if for all $a \neq 0$, $a \in A$, there exists a $b \in A$ such that $ab = ba = 1$.
\end{definition}

\begin{definition}[$GL_n(F)$]
Let $F$ be a field (e.g., $F = \mathbb{R}, \mathbb{Q}, \mathbb{C}, \Z / p\Z $ ($p$ prime)). Then
\begin{equation}
GL_n(F) = \left\{ n \times n \text{ matrices over $F$ with non-zero determinant}\right\}
\end{equation}
\end{definition}

\begin{definition}[Subring]
Let $A$ b a ring. We call $R \subset A$ a subring if the following conditions are satisfied:
\begin{enumerate}
\item Additive and multiplicative identity: $0,1 \in R$.
\item Closed under addition: For all $a,b \in R$, $a+b \in R$.
\item Closed under multiplication: For all $a,b \in R$, $ab \in R$.
\item Closed under inverses (addition): For all $a \in R$, $-a \in R$.
\end{enumerate}
We then write $R \leq A$.
\end{definition}

\begin{definition}[Homomorphism of rings]
Let $A,B$ be rings. A function $\phi: A \to B$ is called a homomorphism of rings if for all $a,b \in A$ the following conditions are satisfied:
\begin{enumerate}
\item $\phi(ab) = \phi(a)\phi(b)$
\item $\phi(a+b) = \phi(a) + \phi(b)$
\item $\phi(1_A) = 1_B$
\end{enumerate}
\end{definition}

\begin{definition}[Kernel of homomorphism of rings]
Suppose $A,B$ are rings and let $\phi:A \to B$ be a ring homomorphism. Then
\begin{equation}
\ker(\phi) = \{a \in A | \phi(a) = 0\} = \phi^{-1}\{0\}
\end{equation}
\end{definition}

\begin{definition}[Ideal]
A subset $I$ of a ring $R$ is called an ideal if the following conditions are satisfied:
\begin{enumerate}
\item Additive identity: $0 \in I$, which assures $I$ is non-empty.
\item Closed under addition: For all $a,b \in I$, $a+b \in I$.
\item Multiplication by elements of ring keeps us in idea: For all $r \in R$ and $a \in I$, $ar, ra \in I$.
\end{enumerate}
\end{definition}

\begin{definition}[Multiplication on $A/I$]
Let $A$ be a ring and $I \subset A$ an ideal. We define multiplication on $A/I$ by
\begin{equation}
(I+a)(I+b) = I + ab
\end{equation}
Further, $A/I$ is a ring.
\end{definition}

\end{document}
